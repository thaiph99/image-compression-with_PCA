%!TEX root = ../book_ML.tex

\chapter{Phương pháp nhân tử Lagrange}
\label{apd:lagrange}
% Trước tiên, chúng ta lại bắt đầu bằng những kỹ thuật đơn giản cho các bài toán cơ bản. Kỹ thuật này có lẽ các bạn đã từng nghe đến: Phương pháp nhân tử Lagrange (method of \href{https://en.wikipedia.org/wiki/Lagrange_multiplier}{Lagrange multipliers}). Đây là một phương pháp giúp tìm các điểm cực trị của hàm mục tiêu trên feasible set của bài toán.

Việc tối ưu hàm số một biến liên tục và khả vi trên miền xác định
là một tập mở\footnote{Xem thêm:
\textit{Open sets, closed
sets and sequences of real numbers} (\url{https://goo.gl/AgKhCn}).} thường được thực hiện dựa trên
việc giải
phương trình đạo hàm bằng không. Gọi hàm mục tiêu là $f(x): \R \rightarrow \R$,
cực trị toàn cục nếu có thường được tìm bằng cách giải
phương trình $f'(x) = 0$. Chú ý rằng điều ngược lại không đúng, tức một điểm
thoả mãn đạo hàm bằng không chưa chắc đã là cực trị của hàm số. Ví dụ hàm $f(x) = x^3$ có đạo hàm bằng không tại $x =0$ nhưng điểm này không là một điểm cực trị. Với hàm nhiều biến, ta cũng có thể áp dụng quan sát này: giải phương trình gradient bằng không.

Cách làm trên đây được áp dụng vào các bài toán tối ưu không ràng buộc. Các bài toán có ràng buộc là một phương
trình:
\begin{eqnarray}
\nonumber
\mathbf{x}=& \arg\min_{\mathbf{x}} f_0(\mathbf{x}) \\\
\label{eqn:18_1constraint}
\text{thoả mãn:}~& f_1(\mathbf{x}) = 0,
\end{eqnarray}
cũng có thể được đưa về bài toán không ràng buộc bằng \textit{phương pháp nhân tử Lagrange}.

\index{phương pháp nhân tử Lagrange}
% Bài toán này là bài toán tổng quát, không nhất thiết phải lồi. Tức hàm mục tiêu và hàm ràng buộc không nhất thiết phải lồi.

Xét hàm số $\mathcal{L}(\mathbf{x}, \lambda) = f_0(\mathbf{x}) + \lambda
f_1(\mathbf{x})$ với biến $\lambda$ được gọi là \textit{nhân tử Lagrange}
({Lagrange multiplier}). Hàm số $\mathcal{L}(\mathbf{x}, \lambda)$ được
gọi là \textit{hàm Lagrange} của bài toán. Người ta đã chứng minh được rằng, điểm tối ưu của
bài toán \eqref{eqn:18_1constraint} thoả mãn điều kiện $\nabla_{\mathbf{x},
\lambda} \mathcal{L}(\mathbf{x}, \lambda) = \bzero$. Điều này tương đương với:
\begin{align}
\label{eqn:18_2}
\nabla_{\bx} \L(\bx, \lambda) &=  \nabla_{\mathbf{x}}f_0(\mathbf{x}) + \lambda
\nabla_{\mathbf{x}} f_1(\mathbf{x}) = 0 \\\
\label{eqn:18_3}
\nabla_{\lambda}\L(\bx, \lambda) &= f_1(\mathbf{x}) = 0
\end{align}

Để ý rằng điều kiện thứ hai chính là ràng buộc trong bài toán
\eqref{eqn:18_1constraint}.

Trong nhiều trường hợp, việc giải hệ phương trình \eqref{eqn:18_2} - \eqref{eqn:18_3} đơn giản hơn việc trực tiếp đi tìm nghiệm của bài toán
\eqref{eqn:18_1constraint}.

\textbf{Ví dụ 1}:

% \subsection{Ví dụ}
Tìm giá trị lớn nhất và nhỏ nhất của hàm số $f_0(x, y) = x + y$ với
$x, y$ thoả mãn điều kiện $f_1(x, y) = x^2 + y^2 = 2$.

% Ta nhận thấy rằng đây không phải là một bài toán tối ưu lồi vì \textit{feasible set} $x^2 + y^2 = 2$ không phải là một tập lồi (nó chỉ là một đường tròn).

\lg

Điều kiện ràng buộc có thể được viết lại dưới dạng $x^2 + y^2 - 2 = 0$.
Hàm Lagrange của bài toán này là $\mathcal{L}(x, y, \lambda) = x + y + \lambda(x^2 + y^2 - 2)$. Các điểm cực trị của hàm số Lagrange phải thoả mãn hệ điều kiện:
\begin{equation}
\label{eqn:18_ex1}
\nabla_{x, y, \lambda} \mathcal{L}(x, y, \lambda) = 0 \Leftrightarrow
\left\{
\begin{matrix}
1 + 2\lambda x &= 0 \\\
1 + 2\lambda y &= 0 \\\
x^2 + y^2 &=     2
\end{matrix}
\right.
\end{equation}
Từ hai phương trình đầu của \eqref{eqn:18_ex1} suy ra $x = y =
\frac{-1}{2\lambda}$. Thay vào phương trình cuối ta sẽ có $\lambda^2 =
\frac{1}{4} \Rightarrow \lambda = \pm \frac{1}{2}$. Vậy ta được 2 cặp nghiệm
$(x, y) \in \{(1, 1), (-1, -1)\}$. Bằng cách thay các giá trị này vào hàm mục
tiêu, ta tìm được giá trị nhỏ nhất và lớn nhất của bài toán.

\textbf{Ví dụ 2}: Chuẩn $\ell_2$ của ma trận.

Nhắc lại chuẩn $\ell_2$ của một vector $\bx: \|\bx\|_2 = \sqrt{\bx^T\bx}$. Dựa trên chuẩn $\ell_2$ của vector, chuẩn $\ell_2$ của một ma trận $\bA \in \R^{m\times n}$, ký hiệu là $\|\bA\|_2$, được định nghĩa như sau:
\begin{equation}
\|\bA\|_2 = \max \frac{\|\bA \bx\|_2}{\|\bx\|_2} = \max \sqrt{\frac{\bx^T\bA^T\bA\bx}{\bx^T\bx}}, \text{với}~ \bx \in \R^{n}
\end{equation}
Bài toán tối ưu này tương đương với:
\begin{eqnarray}
\label{eqn:app_norm2}
\begin{aligned}
\max\left(\bx^T\bA^T\bA\bx\right) \\
\text{thoả mãn:}~ \bx^T\bx = 1
\end{aligned}
\end{eqnarray}
Hàm Lagrange của bài toán này là
\begin{equation}
\L(\bx, \lambda) = \bx^T\bA^T\bA\bx + \lambda ( 1 - \bx^T\bx)
\end{equation}
Các điểm cực trị của hàm số Lagrange phải thoả mãn:
\begin{eqnarray}
\label{eqn:app_9}
\nabla_{\bx}\L &=& 2\bA^T\bA\bx - 2\lambda\bx = \bzero\\
\label{eqn:app_10}
\nabla_{\lambda}\L &=& 1 - \bx^T\bx = 0
\end{eqnarray}
Từ~\eqref{eqn:app_9} ta có $\bA^T\bA\bx = \lambda\bx$. Vậy $\bx$ phải là một vector riêng của $\bA^T\bA$ ứng với trị riêng $\lambda$. Nhân cả hai vế của biểu thức này với $\bx^T$ vào bên trái và sử dụng~\eqref{eqn:app_10}, ta thu được:
\begin{equation}
\bx^T\bA^T\bA\bx = \lambda \bx^T\bx = \lambda
\end{equation}
Từ đó suy ra $\|\bA\bx\|_2$ đạt giá trị lớn nhất khi $\lambda$ đạt giá trị lớn
nhất. Nói cách khác, $\lambda$ phải là trị riêng lớn nhất của $\bA^T\bA$.
Vậy,
\begin{math}
\|\bA\|_2 = \lambda_{\max}(\bA^T\bA).
\end{math}

Các trị riêng của $\bA^T\bA$ còn được gọi là \textit{giá trị suy biến} (singular value) của $\bA$.
Tóm lại, chuẩn $\ell_2$ của một ma trận là giá trị suy biến lớn nhất của ma trận đó.

Hoàn toàn tương tự, nghiệm của bài toán
\begin{equation}
\min_{\|\bx\| = 1} \|\bA\bx\|_2
\end{equation}
chính là một vector riêng ứng với giá trị suy biến nhỏ nhất của $\bA$.
